%%%%%%%%%%%%%%%%%%%%%%%%%%%%%%%%%%%%%%%%%
% Beamer Presentation
% LaTeX Template

\documentclass{beamer}
\mode<presentation> {
\usetheme{Warsaw}
}

\usepackage{multicol}
\usepackage[russian]{babel}
\usepackage{graphicx} 
\usepackage{hyperref}

\title[Introduction to Python]{Files} 
\author{Sugarkhuu Radnaa} 
\institute[]
{
Py4Econ in Ulaanbaatar \\ 
\medskip
\textit{py4econ@gmail.com} 
}
\date{}  % 

\begin{document}

\begin{frame}
\titlepage % Print the title page as the first slide
\end{frame}

\begin{frame}
    \frametitle{Week 4: Learning objectives}
Learn to read, edit and write various file types: 
    \begin{itemize}
        \item Excel
        \item CSV
        \item PDF 
        \item Word
    \end{itemize}
\vskip 1 mm
+
\vskip 1 mm
glob module and text files.
\vskip 2 mm
Not to discuss, but useful file formats: json, html, xml, latex, pickle

\end{frame}

%------------------------------------------------
\section{Different file types} 
%------------------------------------------------

\begin{frame}
    \frametitle{A short about glob module and text files}
            \begin{itemize}
                \item import glob
                \item glob.glob("dir/dir/dir/*.pdf") – pdf files in the "dir/dir/dir/" directory
                
            \end{itemize}
\end{frame}

\begin{frame}
    \frametitle{Excel}
            \begin{itemize}
                \item pandas
                \item openpyxl – read and write xls \textbackslash xlsx
                \item xlrd – read xls \textbackslash xlsx
                \item xlwt - write xls
                \item xlsxwriter
                \item xlwings
            \end{itemize}
\end{frame}

\begin{frame}
    \frametitle{CSV}
            \begin{itemize}
                \item csv
                \item pandas
            \end{itemize}
\end{frame}

\begin{frame}
    \frametitle{PDF}
            \begin{itemize}
                \item pyPDF2 – good for reading and editing
                \item fpdf – good for writing
                \item pdfkit – for writing
                \item tabula-py – working with tables
                \item textract
                \item reportlab
            \end{itemize}
\end{frame}

\begin{frame}
    \frametitle{Word}
            \begin{itemize}
                \item python-docx                
            \end{itemize}
\end{frame}

%------------------------------------------------
\section{Homework} 
%------------------------------------------------

\begin{frame}
    \frametitle{Homework}
    \begin{enumerate}
        \item Task 1
        \item Task 2
        \item Task 3
        \item Task 4
    \end{enumerate}

    \vskip 2mm
    \begin{itemize}
        \item Submit your result as a Github repository
        \item Deadline: 1 week %15 January, 2022
    \end{itemize}

\vfill
\textbf{Note:} Create a github repo from the start and populate it with your results step by step.
\end{frame}

\begin{frame}
    \frametitle{Task 1: Excel}
    Python excel module-уудаар өөрөө зохион жишээ дата үүсгэн 
    (write a file by openpyxl or other packages), түүн дээр тооцсон томьёо, chart-тай эксел файл үүсгэнэ үү.
\end{frame}

\begin{frame}
    \frametitle{Task 2: PDF}
    Энгийн хэд хэдэн "pdf" файл үүсгэн, тэднийг нэгтгэнэ үү. Нэгтгэхдээ glob модулийг ашиглаарай.
\end{frame}

\begin{frame}
    \frametitle{Task 3: Word}
    task3.docx файлыг үүсгэх код зохионо уу. 
\end{frame}

\begin{frame}
    \frametitle{Task 4: CSV}
    \begin{enumerate}
        \item Файлын нэр нь 3 латин үсгээс бүрдсэн байх тус бүр нь ердөө нэг үг бүхий 1000 ширхэг “csv” файлыг нэг фолдерт үүсгэ. 
        \item Эдгээр файлаас a-h хүртэлх үсгээр эхэлсэн файлууд дахь үгсийг нийлүүлсэн нэг файл үүсгэ. 
        \item Нийт хэдэн файлыг 2-р алхамд нэгтгэсэн бэ?
    \end{enumerate}
    
\end{frame}


\begin{frame}
\Huge{\centerline{Thank you!}}
\end{frame}

%----------------------------------------------------------------------------------------

\end{document} 